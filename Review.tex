%%%%%%%%%%%%%%%%%%%%%%%%%%%%%%%%%%%%%%
%%%
% The Legrand Orange Book%1.1 LaTeX Template
% Version 2.
% Th(14/2/16)
%iemplate hass t ebeownloaded fn drom:
% http://www.LaTeXTemplates.com
%
% Original author:
% Mathias Legrand (legrand.mathias@gmail.com) with modifications by:
% Vel (vel@latextemplates.com)
%
% License:
% CC BY-NC-SA 3.0 (http://creativecommons.org/licenses/by-nc-sa/3.0/)
%
% Compiling this template:
% This template uses biber for its bibliography and makeindex for its index.
% When you first open the template, compile it from the command line with the
% commands below to make sure your LaTeX distribution is configured correctly:
%
% 1) pdflatex main
% 2) makeindex main.idx -s StyleInd.ist
% 3) biber main
% 4) pdflatex main x 2
%
% After this, when you wish to update the bibliography/index use the appropriate
% command above and make sure to compile with pdflatex several times
% afterwards to propagate your changes to the document.
%
% This template also uses a number of packages which may need to be
% updated to the newest versions for the template to compile. It is strongly
% recommended you update your LaTeX distribution if you have any
% compilation errors.
%
% Important note:
% Chapter heading images should have a 2:1 width:height ratio,
% e.g. 920px width and 460px height.
%
%%%%%%%%%%%%%%%%%%%%%%%%%%%%%%%%%%%%%%%%%

%----------------------------------------------------------------------------------------
%	PACKAGES AND OTHER DOCUMENT CONFIGURATIONS
%----------------------------------------------------------------------------------------

\documentclass[11pt,fleqn]{book} % Default font size and left-justified equations

%----------------------------------------------------------------------------------------

%%%%%%%%%%%%%%%%%%%%%%%%%%%%%%%%%%%%%%%%%
% The Legrand Orange Book
% Structural Definitions File
% Version 2.0 (9/2/15)
%
% Original author:
% Mathias Legrand (legrand.mathias@gmail.com) with modifications by:
% Vel (vel@latextemplates.com)
%
% This file has been downloaded from:
% http://www.LaTeXTemplates.com
%
% License:
% CC BY-NC-SA 3.0 (http://creativecommons.org/licenses/by-nc-sa/3.0/)
%
%%%%%%%%%%%%%%%%%%%%%%%%%%%%%%%%%%%%%%%%%

%----------------------------------------------------------------------------------------
%	VARIOUS REQUIRED PACKAGES AND CONFIGURATIONS
%----------------------------------------------------------------------------------------

\usepackage[top=3cm,bottom=3cm,left=3cm,right=3cm,headsep=10pt,a4paper]{geometry} % Page margins

\usepackage{graphicx} % Required for including pictures
\graphicspath{{Pictures/}} % Specifies the directory where pictures are stored

\usepackage{lipsum} % Inserts dummy text

\usepackage{tikz} % Required for drawing custom shapes

\usepackage[english]{babel} % English language/hyphenation

\usepackage{enumitem} % Customize lists
\setlist{nolistsep} % Reduce spacing between bullet points and numbered lists

\usepackage{booktabs} % Required for nicer horizontal rules in tables

\usepackage{xcolor} % Required for specifying colors by name
\definecolor{ocre}{RGB}{243,102,25} % Define the orange color used for highlighting throughout the book

%----------------------------------------------------------------------------------------
%	FONTS
%----------------------------------------------------------------------------------------

\usepackage{avant} % Use the Avantgarde font for headings
%\usepackage{times} % Use the Times font for headings
\usepackage{mathptmx} % Use the Adobe Times Roman as the default text font together with math symbols from the Sym­bol, Chancery and Com­puter Modern fonts

\usepackage{microtype} % Slightly tweak font spacing for aesthetics
\usepackage[utf8]{inputenc} % Required for including letters with accents
\usepackage[T1]{fontenc} % Use 8-bit encoding that has 256 glyphs

%----------------------------------------------------------------------------------------
%	BIBLIOGRAPHY AND INDEX
%----------------------------------------------------------------------------------------

\usepackage[style=alphabetic,citestyle=numeric,sorting=nyt,sortcites=true,autopunct=true,babel=hyphen,hyperref=true,abbreviate=false,backref=true,backend=biber]{biblatex}
\addbibresource{bibliography.bib} % BibTeX bibliography file
\defbibheading{bibempty}{}

\usepackage{calc} % For simpler calculation - used for spacing the index letter headings correctly
\usepackage{makeidx} % Required to make an index
\makeindex % Tells LaTeX to create the files required for indexing

%----------------------------------------------------------------------------------------
%	MAIN TABLE OF CONTENTS
%----------------------------------------------------------------------------------------

\usepackage{titletoc} % Required for manipulating the table of contents

\contentsmargin{0cm} % Removes the default margin

% Part text styling
\titlecontents{part}[0cm]
{\addvspace{20pt}\centering\large\bfseries}
{}
{}
{}

% Chapter text styling
\titlecontents{chapter}[1.25cm] % Indentation
{\addvspace{12pt}\large\sffamily\bfseries} % Spacing and font options for chapters
{\color{ocre!60}\contentslabel[\Large\thecontentslabel]{1.25cm}\color{ocre}} % Chapter number
{\color{ocre}}
{\color{ocre!60}\normalsize\;\titlerule*[.5pc]{.}\;\thecontentspage} % Page number

% Section text styling
\titlecontents{section}[1.25cm] % Indentation
{\addvspace{3pt}\sffamily\bfseries} % Spacing and font options for sections
{\contentslabel[\thecontentslabel]{1.25cm}} % Section number
{}
{\hfill\color{black}\thecontentspage} % Page number
[]

% Subsection text styling
\titlecontents{subsection}[1.25cm] % Indentation
{\addvspace{1pt}\sffamily\small} % Spacing and font options for subsections
{\contentslabel[\thecontentslabel]{1.25cm}} % Subsection number
{}
{\ \titlerule*[.5pc]{.}\;\thecontentspage} % Page number
[]

% List of figures
\titlecontents{figure}[0em]
{\addvspace{-5pt}\sffamily}
{\thecontentslabel\hspace*{1em}}
{}
{\ \titlerule*[.5pc]{.}\;\thecontentspage}
[]

% List of tables
\titlecontents{table}[0em]
{\addvspace{-5pt}\sffamily}
{\thecontentslabel\hspace*{1em}}
{}
{\ \titlerule*[.5pc]{.}\;\thecontentspage}
[]

%----------------------------------------------------------------------------------------
%	MINI TABLE OF CONTENTS IN PART HEADS
%----------------------------------------------------------------------------------------

% Chapter text styling
\titlecontents{lchapter}[0em] % Indenting
{\addvspace{15pt}\large\sffamily\bfseries} % Spacing and font options for chapters
{\color{ocre}\contentslabel[\Large\thecontentslabel]{1.25cm}\color{ocre}} % Chapter number
{}
{\color{ocre}\normalsize\sffamily\bfseries\;\titlerule*[.5pc]{.}\;\thecontentspage} % Page number

% Section text styling
\titlecontents{lsection}[0em] % Indenting
{\sffamily\small} % Spacing and font options for sections
{\contentslabel[\thecontentslabel]{1.25cm}} % Section number
{}
{}

% Subsection text styling
\titlecontents{lsubsection}[.5em] % Indentation
{\normalfont\footnotesize\sffamily} % Font settings
{}
{}
{}

%----------------------------------------------------------------------------------------
%	PAGE HEADERS
%----------------------------------------------------------------------------------------

\usepackage{fancyhdr} % Required for header and footer configuration

\pagestyle{fancy}
\renewcommand{\chaptermark}[1]{\markboth{\sffamily\normalsize\bfseries\chaptername\ \thechapter.\ #1}{}} % Chapter text font settings
\renewcommand{\sectionmark}[1]{\markright{\sffamily\normalsize\thesection\hspace{5pt}#1}{}} % Section text font settings
\fancyhf{} \fancyhead[LE,RO]{\sffamily\normalsize\thepage} % Font setting for the page number in the header
\fancyhead[LO]{\rightmark} % Print the nearest section name on the left side of odd pages
\fancyhead[RE]{\leftmark} % Print the current chapter name on the right side of even pages
\renewcommand{\headrulewidth}{0.5pt} % Width of the rule under the header
\addtolength{\headheight}{2.5pt} % Increase the spacing around the header slightly
\renewcommand{\footrulewidth}{0pt} % Removes the rule in the footer
\fancypagestyle{plain}{\fancyhead{}\renewcommand{\headrulewidth}{0pt}} % Style for when a plain pagestyle is specified

% Removes the header from odd empty pages at the end of chapters
\makeatletter
\renewcommand{\cleardoublepage}{
\clearpage\ifodd\c@page\else
\hbox{}
\vspace*{\fill}
\thispagestyle{empty}
\newpage
\fi}

%----------------------------------------------------------------------------------------
%	THEOREM STYLES
%----------------------------------------------------------------------------------------

\usepackage{amsmath,amsfonts,amssymb,amsthm} % For math equations, theorems, symbols, etc

\newcommand{\intoo}[2]{\mathopen{]}#1\,;#2\mathclose{[}}
\newcommand{\ud}{\mathop{\mathrm{{}d}}\mathopen{}}
\newcommand{\intff}[2]{\mathopen{[}#1\,;#2\mathclose{]}}
\newtheorem{notation}{Notation}[chapter]

% Boxed/framed environments
\newtheoremstyle{ocrenumbox}% % Theorem style name
{0pt}% Space above
{0pt}% Space below
{\normalfont}% % Body font
{}% Indent amount
{\small\bf\sffamily\color{ocre}}% % Theorem head font
{\;}% Punctuation after theorem head
{0.25em}% Space after theorem head
{\small\sffamily\color{ocre}\thmname{#1}\nobreakspace\thmnumber{\@ifnotempty{#1}{}\@upn{#2}}% Theorem text (e.g. Theorem 2.1)
\thmnote{\nobreakspace\the\thm@notefont\sffamily\bfseries\color{black}---\nobreakspace#3.}} % Optional theorem note
\renewcommand{\qedsymbol}{$\blacksquare$}% Optional qed square

\newtheoremstyle{blacknumex}% Theorem style name
{5pt}% Space above
{5pt}% Space below
{\normalfont}% Body font
{} % Indent amount
{\small\bf\sffamily}% Theorem head font
{\;}% Punctuation after theorem head
{0.25em}% Space after theorem head
{\small\sffamily{\tiny\ensuremath{\blacksquare}}\nobreakspace\thmname{#1}\nobreakspace\thmnumber{\@ifnotempty{#1}{}\@upn{#2}}% Theorem text (e.g. Theorem 2.1)
\thmnote{\nobreakspace\the\thm@notefont\sffamily\bfseries---\nobreakspace#3.}}% Optional theorem note

\newtheoremstyle{blacknumbox} % Theorem style name
{0pt}% Space above
{0pt}% Space below
{\normalfont}% Body font
{}% Indent amount
{\small\bf\sffamily}% Theorem head font
{\;}% Punctuation after theorem head
{0.25em}% Space after theorem head
{\small\sffamily\thmname{#1}\nobreakspace\thmnumber{\@ifnotempty{#1}{}\@upn{#2}}% Theorem text (e.g. Theorem 2.1)
\thmnote{\nobreakspace\the\thm@notefont\sffamily\bfseries---\nobreakspace#3.}}% Optional theorem note

% Non-boxed/non-framed environments
\newtheoremstyle{ocrenum}% % Theorem style name
{5pt}% Space above
{5pt}% Space below
{\normalfont}% % Body font
{}% Indent amount
{\small\bf\sffamily\color{ocre}}% % Theorem head font
{\;}% Punctuation after theorem head
{0.25em}% Space after theorem head
{\small\sffamily\color{ocre}\thmname{#1}\nobreakspace\thmnumber{\@ifnotempty{#1}{}\@upn{#2}}% Theorem text (e.g. Theorem 2.1)
\thmnote{\nobreakspace\the\thm@notefont\sffamily\bfseries\color{black}---\nobreakspace#3.}} % Optional theorem note
\renewcommand{\qedsymbol}{$\blacksquare$}% Optional qed square
\makeatother

% Defines the theorem text style for each type of theorem to one of the three styles above
\newcounter{dummy}
\numberwithin{dummy}{section}
\theoremstyle{ocrenumbox}
\newtheorem{theoremeT}[dummy]{Theorem}
\newtheorem{problem}{Problem}[chapter]
\newtheorem{exerciseT}{Exercise}[chapter]
\theoremstyle{blacknumex}
\newtheorem{exampleT}{Example}[chapter]
\theoremstyle{blacknumbox}
\newtheorem{vocabulary}{Vocabulary}[chapter]
\newtheorem{definitionT}{Definition}[section]
\newtheorem{corollaryT}[dummy]{Corollary}
\theoremstyle{ocrenum}
\newtheorem{proposition}[dummy]{Proposition}

%----------------------------------------------------------------------------------------
%	DEFINITION OF COLORED BOXES
%----------------------------------------------------------------------------------------

\RequirePackage[framemethod=default]{mdframed} % Required for creating the theorem, definition, exercise and corollary boxes

% Theorem box
\newmdenv[skipabove=7pt,
skipbelow=7pt,
backgroundcolor=black!5,
linecolor=ocre,
innerleftmargin=5pt,
innerrightmargin=5pt,
innertopmargin=5pt,
leftmargin=0cm,
rightmargin=0cm,
innerbottommargin=5pt]{tBox}

% Exercise box
\newmdenv[skipabove=7pt,
skipbelow=7pt,
rightline=false,
leftline=true,
topline=false,
bottomline=false,
backgroundcolor=ocre!10,
linecolor=ocre,
innerleftmargin=5pt,
innerrightmargin=5pt,
innertopmargin=5pt,
innerbottommargin=5pt,
leftmargin=0cm,
rightmargin=0cm,
linewidth=4pt]{eBox}

% Definition box
\newmdenv[skipabove=7pt,
skipbelow=7pt,
rightline=false,
leftline=true,
topline=false,
bottomline=false,
linecolor=ocre,
innerleftmargin=5pt,
innerrightmargin=5pt,
innertopmargin=0pt,
leftmargin=0cm,
rightmargin=0cm,
linewidth=4pt,
innerbottommargin=0pt]{dBox}

% Corollary box
\newmdenv[skipabove=7pt,
skipbelow=7pt,
rightline=false,
leftline=true,
topline=false,
bottomline=false,
linecolor=gray,
backgroundcolor=black!5,
innerleftmargin=5pt,
innerrightmargin=5pt,
innertopmargin=5pt,
leftmargin=0cm,
rightmargin=0cm,
linewidth=4pt,
innerbottommargin=5pt]{cBox}

% Creates an environment for each type of theorem and assigns it a theorem text style from the "Theorem Styles" section above and a colored box from above
\newenvironment{theorem}{\begin{tBox}\begin{theoremeT}}{\end{theoremeT}\end{tBox}}
\newenvironment{exercise}{\begin{eBox}\begin{exerciseT}}{\hfill{\color{ocre}\tiny\ensuremath{\blacksquare}}\end{exerciseT}\end{eBox}}
\newenvironment{definition}{\begin{dBox}\begin{definitionT}}{\end{definitionT}\end{dBox}}
\newenvironment{example}{\begin{exampleT}}{\hfill{\tiny\ensuremath{\blacksquare}}\end{exampleT}}
\newenvironment{corollary}{\begin{cBox}\begin{corollaryT}}{\end{corollaryT}\end{cBox}}

%----------------------------------------------------------------------------------------
%	REMARK ENVIRONMENT
%----------------------------------------------------------------------------------------

\newenvironment{remark}{\par\vspace{10pt}\small % Vertical white space above the remark and smaller font size
\begin{list}{}{
\leftmargin=35pt % Indentation on the left
\rightmargin=25pt}\item\ignorespaces % Indentation on the right
\makebox[-2.5pt]{\begin{tikzpicture}[overlay]
\node[draw=ocre!60,line width=1pt,circle,fill=ocre!25,font=\sffamily\bfseries,inner sep=2pt,outer sep=0pt] at (-15pt,0pt){\textcolor{ocre}{R}};\end{tikzpicture}} % Orange R in a circle
\advance\baselineskip -1pt}{\end{list}\vskip5pt} % Tighter line spacing and white space after remark

%----------------------------------------------------------------------------------------
%	SECTION NUMBERING IN THE MARGIN
%----------------------------------------------------------------------------------------

\makeatletter
\renewcommand{\@seccntformat}[1]{\llap{\textcolor{ocre}{\csname the#1\endcsname}\hspace{1em}}}
\renewcommand{\section}{\@startsection{section}{1}{\z@}
{-4ex \@plus -1ex \@minus -.4ex}
{1ex \@plus.2ex }
{\normalfont\large\sffamily\bfseries}}
\renewcommand{\subsection}{\@startsection {subsection}{2}{\z@}
{-3ex \@plus -0.1ex \@minus -.4ex}
{0.5ex \@plus.2ex }
{\normalfont\sffamily\bfseries}}
\renewcommand{\subsubsection}{\@startsection {subsubsection}{3}{\z@}
{-2ex \@plus -0.1ex \@minus -.2ex}
{.2ex \@plus.2ex }
{\normalfont\small\sffamily\bfseries}}
\renewcommand\paragraph{\@startsection{paragraph}{4}{\z@}
{-2ex \@plus-.2ex \@minus .2ex}
{.1ex}
{\normalfont\small\sffamily\bfseries}}

%----------------------------------------------------------------------------------------
%	PART HEADINGS
%----------------------------------------------------------------------------------------

% numbered part in the table of contents
\newcommand{\@mypartnumtocformat}[2]{%
\setlength\fboxsep{0pt}%
\noindent\colorbox{ocre!20}{\strut\parbox[c][.7cm]{\ecart}{\color{ocre!70}\Large\sffamily\bfseries\centering#1}}\hskip\esp\colorbox{ocre!40}{\strut\parbox[c][.7cm]{\linewidth-\ecart-\esp}{\Large\sffamily\centering#2}}}%
%%%%%%%%%%%%%%%%%%%%%%%%%%%%%%%%%%
% unnumbered part in the table of contents
\newcommand{\@myparttocformat}[1]{%
\setlength\fboxsep{0pt}%
\noindent\colorbox{ocre!40}{\strut\parbox[c][.7cm]{\linewidth}{\Large\sffamily\centering#1}}}%
%%%%%%%%%%%%%%%%%%%%%%%%%%%%%%%%%%
\newlength\esp
\setlength\esp{4pt}
\newlength\ecart
\setlength\ecart{1.2cm-\esp}
\newcommand{\thepartimage}{}%
\newcommand{\partimage}[1]{\renewcommand{\thepartimage}{#1}}%
\def\@part[#1]#2{%
\ifnum \c@secnumdepth >-2\relax%
\refstepcounter{part}%
\addcontentsline{toc}{part}{\texorpdfstring{\protect\@mypartnumtocformat{\thepart}{#1}}{\partname~\thepart\ ---\ #1}}
\else%
\addcontentsline{toc}{part}{\texorpdfstring{\protect\@myparttocformat{#1}}{#1}}%
\fi%
\startcontents%
\markboth{}{}%
{\thispagestyle{empty}%
\begin{tikzpicture}[remember picture,overlay]%
\node at (current page.north west){\begin{tikzpicture}[remember picture,overlay]%
\fill[ocre!20](0cm,0cm) rectangle (\paperwidth,-\paperheight);
\node[anchor=north] at (4cm,-3.25cm){\color{ocre!40}\fontsize{220}{100}\sffamily\bfseries\@Roman\c@part};
\node[anchor=south east] at (\paperwidth-1cm,-\paperheight+1cm){\parbox[t][][t]{8.5cm}{
\printcontents{l}{0}{\setcounter{tocdepth}{1}}%
}};
\node[anchor=north east] at (\paperwidth-1.5cm,-3.25cm){\parbox[t][][t]{15cm}{\strut\raggedleft\color{white}\fontsize{30}{30}\sffamily\bfseries#2}};
\end{tikzpicture}};
\end{tikzpicture}}%
\@endpart}
\def\@spart#1{%
\startcontents%
\phantomsection
{\thispagestyle{empty}%
\begin{tikzpicture}[remember picture,overlay]%
\node at (current page.north west){\begin{tikzpicture}[remember picture,overlay]%
\fill[ocre!20](0cm,0cm) rectangle (\paperwidth,-\paperheight);
\node[anchor=north east] at (\paperwidth-1.5cm,-3.25cm){\parbox[t][][t]{15cm}{\strut\raggedleft\color{white}\fontsize{30}{30}\sffamily\bfseries#1}};
\end{tikzpicture}};
\end{tikzpicture}}
\addcontentsline{toc}{part}{\texorpdfstring{%
\setlength\fboxsep{0pt}%
\noindent\protect\colorbox{ocre!40}{\strut\protect\parbox[c][.7cm]{\linewidth}{\Large\sffamily\protect\centering #1\quad\mbox{}}}}{#1}}%
\@endpart}
\def\@endpart{\vfil\newpage
\if@twoside
\if@openright
\null
\thispagestyle{empty}%
\newpage
\fi
\fi
\if@tempswa
\twocolumn
\fi}

%----------------------------------------------------------------------------------------
%	CHAPTER HEADINGS
%----------------------------------------------------------------------------------------

% A switch to conditionally include a picture, implemented by  Christian Hupfer
\newif\ifusechapterimage
\usechapterimagetrue
\newcommand{\thechapterimage}{}%
\newcommand{\chapterimage}[1]{\ifusechapterimage\renewcommand{\thechapterimage}{#1}\fi}%
\def\@makechapterhead#1{%
{\parindent \z@ \raggedright \normalfont
\ifnum \c@secnumdepth >\m@ne
\if@mainmatter
\begin{tikzpicture}[remember picture,overlay]
\node at (current page.north west)
{\begin{tikzpicture}[remember picture,overlay]
\node[anchor=north west,inner sep=0pt] at (0,0) {\ifusechapterimage\includegraphics[width=\paperwidth]{\thechapterimage}\fi};
\draw[anchor=west] (\Gm@lmargin,-5cm) node [line width=2pt,rounded corners=15pt,draw=ocre,fill=white,fill opacity=0.5,inner sep=15pt]{\strut\makebox[22cm]{}};
\draw[anchor=west] (\Gm@lmargin+.3cm,-5cm) node {\huge\sffamily\bfseries\color{black}\thechapter. #1\strut};
\end{tikzpicture}};
\end{tikzpicture}
\else
\begin{tikzpicture}[remember picture,overlay]
\node at (current page.north west)
{\begin{tikzpicture}[remember picture,overlay]
\node[anchor=north west,inner sep=0pt] at (0,0) {\ifusechapterimage\includegraphics[width=\paperwidth]{\thechapterimage}\fi};
\draw[anchor=west] (\Gm@lmargin,-5cm) node [line width=2pt,rounded corners=15pt,draw=ocre,fill=white,fill opacity=0.5,inner sep=15pt]{\strut\makebox[22cm]{}};
\draw[anchor=west] (\Gm@lmargin+.3cm,-5cm) node {\huge\sffamily\bfseries\color{black}#1\strut};
\end{tikzpicture}};
\end{tikzpicture}
\fi\fi\par\vspace*{135\p@}}}

%-------------------------------------------

\def\@makeschapterhead#1{%
\begin{tikzpicture}[remember picture,overlay]
\node at (current page.north west)
{\begin{tikzpicture}[remember picture,overlay]
\node[anchor=north west,inner sep=0pt] at (0,0) {\ifusechapterimage\includegraphics[width=\paperwidth]{\thechapterimage}\fi};
\draw[anchor=west] (\Gm@lmargin,-5cm) node [line width=2pt,rounded corners=15pt,draw=ocre,fill=white,fill opacity=0.5,inner sep=15pt]{\strut\makebox[22cm]{}};
\draw[anchor=west] (\Gm@lmargin+.3cm,-5cm) node {\huge\sffamily\bfseries\color{black}#1\strut};
\end{tikzpicture}};
\end{tikzpicture}
\par\vspace*{135\p@}}
\makeatother

%----------------------------------------------------------------------------------------
%	HYPERLINKS IN THE DOCUMENTS
%----------------------------------------------------------------------------------------

\usepackage{hyperref}
\hypersetup{hidelinks,backref=true,pagebackref=true,hyperindex=true,colorlinks=false,breaklinks=true,urlcolor= ocre,bookmarks=true,bookmarksopen=false,pdftitle={Title},pdfauthor={Author}}
\usepackage{bookmark}
\bookmarksetup{
open,
numbered,
addtohook={%
\ifnum\bookmarkget{level}=0 % chapter
\bookmarksetup{bold}%
\fi
\ifnum\bookmarkget{level}=-1 % part
\bookmarksetup{color=ocre,bold}%
\fi
}
}
 % Insert the commands.tex file which contains the majority of the structure behind the template

\usepackage{ulem}
\usepackage{physics}
\setlength{\intextsep}{.5ex}

\colorlet{ocre}{black} % Define the orange color used for highlighting throughout the book
\definecolor{darkgreen}{RGB}{0,128,0}

\newmdenv[skipabove=7pt,
skipbelow=7pt,
rightline=false,
leftline=true,
topline=false,
bottomline=false,
linecolor=red,
backgroundcolor=red!5,
innerleftmargin=5pt,
innerrightmargin=5pt,
innertopmargin=5pt,
leftmargin=0cm,
rightmargin=0cm,
linewidth=4pt,
innerbottommargin=5pt]{rBox}

\renewmdenv[skipabove=7pt,
skipbelow=7pt,
rightline=false,
leftline=true,
topline=false,
bottomline=false,
backgroundcolor=darkgreen!10,
linecolor=darkgreen,
innerleftmargin=5pt,
innerrightmargin=5pt,
innertopmargin=5pt,
innerbottommargin=5pt,
leftmargin=0cm,
rightmargin=0cm,
linewidth=4pt]{eBox}

\newenvironment{emphbox}{\begin{rBox}}{\end{rBox}}

\renewenvironment{remark}{\par\vspace{10pt}\small % Vertical white space above the remark and smaller font size
\begin{list}{}{
\leftmargin=35pt % Indentation on the left
\rightmargin=25pt}\item\ignorespaces % Indentation on the right
\makebox[-2.5pt]{\begin{tikzpicture}[overlay]
\node[draw=blue!60,line width=1pt,circle,fill=blue!25,font=\sffamily\bfseries,inner sep=2pt,outer sep=0pt] at (-15pt,0pt){\textcolor{blue}{R}};\end{tikzpicture}} % Orange R in a circle
\advance\baselineskip -1pt}{\end{list}\vskip5pt} % Tighter line spacing and white space after remark % Insert the custom_style.tex file which contains the
                     % custom part of the structure behind the template
\begin{document}

%----------------------------------------------------------------------------------------
%	TITLE PAGE
%----------------------------------------------------------------------------------------

\begingroup
\thispagestyle{empty}
\begin{tikzpicture}[remember picture,overlay]
\coordinate [below=12cm] (midpoint) at (current page.north);
\node at (current page.north west)
{\begin{tikzpicture}[remember picture,overlay]
    \node[anchor=north west,inner sep=0pt] at (0,0) {\includegraphics[width=\paperwidth]{Pictures/cover}}; % Background image
\draw[anchor=north] (midpoint) node [fill=ocre!30!white,fill opacity=0.6,text opacity=1,inner sep=1cm]{\Huge\centering\bfseries\sffamily\parbox[c][][t]{\paperwidth}{\centering General Relativity\\[15pt] % Book title
{\Large Notes Review}\\[20pt] % Subtitle
{\huge Yidun Wan}}}; % Author name
\end{tikzpicture}};
\end{tikzpicture}
\vfill
\endgroup

%----------------------------------------------------------------------------------------
%	COPYRIGHT PAGE
%----------------------------------------------------------------------------------------

\newpage
~\vfill
\thispagestyle{empty}

\noindent Copyright \copyright\ 2017 Yidun Wan\\ % Copyright notice

\noindent \textsc{Organized by kenkangxgwe}\\ % Publisher

\noindent \textsc{github.com/kenkangxgwe/course-notes}\\ % URL

\noindent Licensed under the Creative Commons Attribution-NonCommercial 3.0 Unported License (the ``License''). You may not use this file except in compliance with the License. You may obtain a copy of the License at \url{http://creativecommons.org/licenses/by-nc/3.0}. Unless required by applicable law or agreed to in writing, software distributed under the License is distributed on an \textsc{``as is'' basis, without warranties or conditions of any kind}, either express or implied. See the License for the specific language governing permissions and limitations under the License.\\ % License information

\noindent \textit{First edition, Spring 2017} % Printing/edition date

%----------------------------------------------------------------------------------------
%	TABLE OF CONTENTS
%----------------------------------------------------------------------------------------

%\usechapterimagefalse % If you don't want to include a chapter image, use this to toggle images off - it can be enabled later with \usechapterimagetrue

\usechapterimagefalse
% \chapterimage{nasa-small} % Table of contents heading image

\pagestyle{empty} % No headers

\tableofcontents % Print the table of contents itself

\cleardoublepage % Forces the first chapter to start on an odd page so it's on the right

\pagestyle{fancy} % Print headers again

%----------------------------------------------------------------------------------------
%	PART
%----------------------------------------------------------------------------------------

\part{Part One}

%----------------------------------------------------------------------------------------
%	CHAPTER 1
%----------------------------------------------------------------------------------------
\abovedisplayskip=.5ex
\belowdisplayskip=.5ex
%\chapterimage{chapter_head_2.pdf} % Chapter heading image
\chapter{Overview and Basics}

\section{Equivalence Principal : a starter}\index{Equivalence Principal}
What is the Equivalence Principal?
\begin{emphbox}
\begin{enumerate}[label=\textcolor{red}{\textcircled{\tiny{\arabic*}}}, leftmargin=*]
\item Locally, gravitational field is equivalent to acceleration.\\
  OR
\item Locally, spacetime is flat.
\end{enumerate}
\end{emphbox}
We explain \textcolor{red}{\textcircled{\tiny{1}}} by thought experiments,
\textcolor{red}{\textcircled{\tiny{2}}} later.


\subsection{Freely-falling observer sees no gravity}
\begin{figure}[h]
\centering\includestandalone[scale=.8]{Pictures/Free-falling-observer}
\end{figure}
\begin{alignat*}{2}
  y' = y + h - \frac{1}{2}gt^2 &\Rightarrow dd ot{y}' &&= dd ot{y} - g \hfill  \\
                               &\Rightarrow dd ot{y}  &&= dd ot{y}' + g \hfill \\
                               &\                     &&= -g + g = 0 \hfill
\end{alignat*}
\begin{emphbox}
  The observer in the elevator sees no gravity.
\end{emphbox}
\begin{remark}
  The derivation is from our pesperctive to show that the observer in the
  elevator indeed sees no gravity. \\
  Nevertheless, the observer himself simply feels no gravity because hee sees
  himself floating along with other objects in the elevator.
\end{remark}
Realizing this was in fact the happiest thought of Einstein, which led him to
discover GR eventually.

\subsection{Accelerating observer feels gravity (fictions)}
\begin{figure}[h]
\centering\includestandalone[scale=.8]{Pictures/Accelerating-observer}
\end{figure}
\begin{alignat*}{2}
  y' = y + \frac{1}{2}gt^2 & \Rightarrow dd ot{y}' && = dd ot{y} + g \\
                           & \Rightarrow dd ot{y}  && = y' -g        \\
                           & \                     && = 0-g = -g
\end{alignat*}
\begin{emphbox}
  The inside observer sees the ball falling to the floor and concludes there is
  gravity acting on the ball.
\end{emphbox}
\begin{remark}
  Of course the inside observer feels being attached to the floor too, in the
  same way he feels on the Earth.
  The outside observer however sees the rocket's floor rushing to a floating ball.
\end{remark}
  Now you see that the centrifugal force is kinda ``gravity''.

\subsection{The equivalence principle is local}
That is, locally gravity can be \uline{transformed} away, or ``created'' by
transformations (acceleration).
The transformation must be nonlinear.
Consider a rocket at uniform speed as a counter example.
\begin{exercise}
  Think about a free particle in an Euclidean plane. Try to transform the
  coordinates nonlinearly to any other coordinate system and see what you can get.
\end{exercise}

\subsection{But not globally}
\begin{figure}[h]
\centering\includestandalone[width=\linewidth]{Pictures/But-not-globally}
\end{figure}
Being unable to transform away gravity in a large region, i.e.\ globally, implies
that something local pertaining to \textcolor{red}{\uppercase{true}} gravity.
cannot be transformed away because a global effect is an aggregation of local ones.
\begin{emphbox}
  Indeed! What cannot be transformed away locally is the \uwave{spacetime curvature},
  which is the source of the tidal force.
\end{emphbox}
\begin{remark}
  This also reflects the second form of the equivalence principle. But we shall
  leave this for later.
\end{remark}

\subsection{Two implications of the E.P.\ }
\begin{emphbox}
  \begin{itemize}[label={--},leftmargin=*]
  \item Gravity acts on everything uniformly.
  \item Inertial mass = gravitational mass.
  \end{itemize}
\end{emphbox}
%------------------------------------------------

\section{Rotation in the Euclidean Plain}
As I said, Einstein gravity and even modern theoretical physics is largely about
invariance. But invariance under what and of what? In the rest of today's
lecture, I shall introduce to you the two ingredients of invariances. Namely
linear transformations and tensors. I shall consider only rotations as an
example of linear transformations here. The linear transformations we will
encounter in this course are merely generalizations of rotations, which will be
explained upon their arrival.

\subsection{Rotation by elementary geometry}
\begin{figure}[h]
\centering\includestandalone[]{Pictures/rotation-elementary}
\end{figure}
\begin{align*}
  \Rightarrow \mqty(x' \\ y) &= \mqty(\cos \theta & \sin \theta \\ -\sin \theta & \cos \theta\\) \mqty(x \\ y) \\
  v' &= R(\theta) v \\
  \text{more explicitly,\quad} v'^i &= R(\theta)^{ij} v^j
\end{align*}
\(v\) is a \uwave{vector} as it transforms as a vector under rotation.
\begin{remark}
  This method can hardly be generalized to higher dimensions. Need a more
  systematic way!
\end{remark}

\subsection{Infinitesimal rotation (generators)}\index{Infinitesimal Rotation}

\subsubsection{Properties of \(R(\theta)\)}
\begin{align*}
  \abs{v'} = \abs{v} &\Rightarrow v^{\mathrm{T}} R^{\mathrm{T}} R V = V^{\mathrm{T}} V \\
                     &\Rightarrow R^{\mathrm{T}} R = 1 \quad \text{\textcolor{red}{orthogonal matrix}}\\
                     &\Rightarrow \det R = \pm 1
\end{align*}
But \(\det R = -1\) describes \uwave{reflections}. E.g.\ \(R=\mqty(\dmat{1, -1})\) is a reflection about the y-axis. \\
\(\Rightarrow\) rotations satisfy \(\det R = 1\) \quad \textcolor{red}{special orthogonal}

\subsubsection{Infinitesimal rotation}
  \(\because \det R = 1\) \\
  \(\therefore\) we can expand \(R = 1 + A + \cdots\) \quad higher order terms negligible
\begin{align*}
  \text{Then,}       &\quad R^{\mathrm{T}} R = 1                                                  \\
                     &\Rightarrow (1+ A^{\mathrm{T}} + \cdots) (1 + A + \cdots) = 1               \\
                     &\Rightarrow 1+A^{\mathrm{T}} + A + \cdots = 1                               \\
                     &\Rightarrow A^{\mathrm{T}} = -A \quad \text{\textcolor{red}{antisymmetric}} \\
  \text{But in 2-d,} &\ \text{any antisymmetric matrix is} \propto J = \mqty(\admat{1,-1})        \\
                     &\Rightarrow A = \theta J                                                    \\
                     &\Rightarrow R(\theta) = 1 + \theta J + \order{\theta^2}                     \\
                     &\Rightarrow \text{for tiny \(\theta\),} \mqty(w' \\ y') = \mqty(x + \theta y \\ -\theta x + y) \quad \text{consistent with Taylor expanding}
\end{align*}
\(\quad x' = \cos \theta x + \sin \theta y\) \quad \( y' = -\sin \theta x + \cos \theta
y\)\\
\(\Rightarrow\) we can restore the rotation \(R(\theta)\) for finite \(\theta\)
by exponentiating \(J\):
\begin{align*}
  R(\theta) = \lim_{N \to \infty} R\left( \frac{\theta}{N} \right)^N &= \lim_{N \to \infty} \left( 1 + \frac{\theta J}{N} \right)^N\\
            \Rightarrow R(\theta)                                    &= \ee^{\theta J} \\
                                                                     &\Downarrow \text{Taylor expansion} \\
                                                                     &= \mqty(\cos \theta & \sin \theta \\ -\sin \theta & \cos \theta)
\end{align*}
\begin{remark}
  \(R(\theta)\) forms the group \(\mathrm{SO}(2) \simeq \mathrm{U}(1)\) \\
  \(A(\theta)\) forms the Lie algebra \(\mathrm{so}(2)\)
\end{remark}
% ------------------------------------------------

\section{Rotation in D-dimension}
\begin{align*}
  &R^{\mathrm{T}} R = 1,            \quad \det R = 1, \quad \\
  &\Rightarrow A^{\mathrm{T}} = -A, \quad R = 1 + A + \cdots \quad \text{hold in any dimensions}
\end{align*}

\subsection{3-dimension}
Obviously, any \(3 \times 3\) antisymmetric matrix \(A\) is a linear combination
of
\begin{align*}
  &J_x=\mqty(\dmat{0,0&1\\-1&0}), \quad J_y = \mqty(\admat{-1,0,1}), \quad J_z = \mqty(\dmat{0&1\\-1&0,0}) \\
  &\text{that is,}                \quad A = \theta_x J_x + \theta_y J_y + \theta_z J_z
\end{align*}
\begin{emphbox}
  Caution: \(J_i J_j \neq J_j J_i\)      \\
  e.g.\    \(J_x J_y - J_y J_x = - J_z\) \\
  Rotation about different axis may not commute.
\end{emphbox}
This case, the Lie group \(\mathrm{SO}(3)\) and Lie algebra \(\mathrm{so}(3)\) are
called non-Abelian.

\subsection{D-dimension}
\begin{exercise}
  I urge you to try the generalization in D-dim. Or you can read the appendix 2
  of Chapter I.3. In later lectures, I'd assume you have done this.
\end{exercise}

\section{Vector and Vector field}\index{Vector}\index{Vector!Vector Field}

\subsection{Vector \& metric}\index{Vector!Metric}
\begin{itemize}[label={--}, leftmargin=*]
\item Common confusion: a point in a D-dim space is \uppercase{not} a vector! Only the
  difference between two points is a vector! \\
  We commonly write \(\vec{v} = (x^1, x^2, x^3 ,\dots, x^D)^{\mathrm{T}}\) as a
  vector but this is done by assuming that \(v\) stems from the origin. \\
  It's better to write \(\vec{v} = \mqty(v^1\\v^2\\v^3\\\vdots\\v^D)\) as a column.
\item Moreover, the infinitesimal difference between two points
  \(\mqty(\dd x^1 \\ \dd x^2 \\ \vdots \\ \dd x^D)\) is
  certainly a vector.
\item In the Euclidean space, the inner-product of two vectors
  \(\vec{u}\),\(\vec{v}\) is \[\vec{u} \cdot \vec{v} = \sum_i u^i v^i =
    \sum_{i,j} \delta^{ij} u^i v^j = \delta^ij u^i v^j \quad \text{\textcolor{red}{Einstein
    summation}}\]
  \(\delta^{ij}\) is called the \uline{metric} of the space. \\
  \begin{remark}
    More general metric will be introduced.
  \end{remark}
  The length\(^2\) of an infinitesimal vector is then
  \[\dd s^2 = \delta^{ij} dd x^i dd x^j\]
  The orthogonality of \(R(\theta)\) becomes
  \[\left(R^{\mathrm{T}}\right)^{ij} R^{jk} = \delta^{ik} = R^{ji} R^{jk}\]
\item A vector in the Euclidean space \uline{transforms linearly under rotation}.
  \[v^{ij} = R^{ij} v^j\]
  Two vectors \(\vec{u}\) and \(\vec{v}\), \(\vec{u} \pm \vec{v}\) is a vector.
  \(\vec{u} \cross \vec{v} = \mqty(u^2 v^3 - u^3 v^2 \\ u^3 v^1 - u^1 v^3 \\ u^1
  v^2 - u^2 v^1)\) is a vector.
  \begin{problem}
    What about \(\mqty(u^2 v^3 \\ u^3 v^1 \\ u^1 v^2)\) ?
  \end{problem}
\item Vector field \\
  A vector field \(\vec{v}(x)\) is a vector-valued function of the space. That
  is, at each point \(x\), \(v(x)\) is a vector:
  \(\vec{v}(x) = \mqty(v^1(x) \\ v^2(x) \\ \vdots \\ v^D(x)).\)
\end{itemize}

\subsection{Newtonian physics is invariant under rotation}
\begin{align*}
  \vec{F}           &= m \vec{a}           \\
                    &\Downarrow            \\
  R(\theta) \vec{F} &= m R(\theta) \vec{a} \\
  \vec{F}'          &= m \vec{a}'
\end{align*}
\begin{emphbox}
  Caution: This is in fact a covariance, i.e., both sides of the equation
  transform in the same way! \\
  What's invariant is the Newton's 2nd law of physics: acceleration is always
  proportional to the force, and the proportionality factor is \(\frac{1}{m}\)!
\end{emphbox}
The above is true only if \(m\) doesn't change under \(R(\theta)\). We say \(m\)
is a \uline{scalar}. \\
Hence, scalars are \uline{invariant} under rotation!
\begin{remark}
  modern physics is keen in looking for invariants under transformations. This
  is the essence of the \uline{action principle}.
\end{remark}

\section{Tensor and tensor field}\index{Tensor}\index{Tensor!Tensor Field}
Scalars can be made of vectors.
E.g.\ the length\(^2\) \(dd s^2 = \delta^{ij} dd x^x
dd x^j\) is a scalar. \\
But not all interesting scalars can be made of just vectors. More complicated
objects are needed. Among these objects here we focus on those transform nicely
under rotations, which are called \uline{tensors}.
\begin{emphbox}
  Slogan: A tensor is something that transforms like a tensor.
\end{emphbox}
\begin{remark}
  Actually, tensors arise naturally. A vector is a spacial case of a tensor.
  Naively, we can put two vectors \(\vec{n}\), \(\vec{v}\) side by side to make a
  2-indexed object.
  \begin{alignat*}{2}
    & T^{ij} &&= u^i v^j \\
    \Rightarrow & T'^{i'j'} &&= R^{i'i} u^i R^{j'j} v^j = R^{i'i} R^{j'j} T^{ij}
  \end{alignat*}
  Each index transforms under \uppercase{one} \(R(\theta)\). \textcolor{red}{This is the meaning of
  ``transforms like a tensor.''}
\end{remark}
But more generally, \(T^{ij}\) is not made of two vectors side by side
\(T^{ij}\) is a rank-2 tensor. There can be rank-n tensors, \(T^{ijk\dots}\).
\[T'^{i'j'k'\dots} = \left(R^{i'i} R^{j'j} R^{k'k} \cdots\right) T^{ijk\dots}.\]

\subsection{Tensor and Representation Theory}\index{Tensor!Representation}
A rotation \(R(\theta)\) is an abstract operation, writing \(R(\theta)\) as a
matrix \(\mathcal{D}(R)\) is a way of representing the action of \(R\). All such matrices
form a representation of the rotation group (special orthogonal group).
\begin{example}
  \(D = 3\), the matrices \(\ee^{\theta_x J_x + \theta_y J_y + \theta_z
    J_z} = \mathcal{D}(R), \quad \forall \theta_x, \theta_y, \theta_z \in
  \mathbb{R}\), form a 3-dimensional \uline{representation} of
  \(\textrm{SO}(3)\), in the sense that \(\mathcal{D}(R_1) \mathcal{D}(R_2) =
  \mathcal{D}(R_1 R_2)\) \\
  All 3-dimensional vectors \(v = \mqty(v^1 \\ v^2 \\
  v^3)\) form the 3-d \uline{representation space} correspondingly. \\
  3-d representation of \(\textrm{SO}(3)\) is the \uwave{fundamental} representation.
\end{example}
But \(\mathrm{so}(3)\) can have higher dimensional representations too.
These are the \uwave{tensor representations}.\\
Consider the rank-2 \(T^{ij}\), \(i,j,=1,2,3\). \\
Arrange all 9 elements in a column array \(\mqty(T^{11} \\ T^{12} \\ T^{13} \\
T^{21} \\ \vdots \\ \vdots \\ T^{33})\)
\begin{align*}
  T^{ij} \to T'^{i'j'} &= \underbrace{R^{i'i} R^{j'j}}_{\text{tensor product}} T^{ij} \\
  &= \underbrace{\tilde{R}^{(i'j')(ij)}}_{9 \times 9\ \text{matrix}} T^{ij}
\end{align*}
\(\Rightarrow T^{ij}\) forms a 9-d representations of \(\mathrm{SO}(3)\).
\begin{problem}
  Is \(T^{ij}\) reducible?
\end{problem}

\subsection{Irreducible representations/tensor}
\begin{itemize}
\item \(T^{ij}\) has 9 elements. \\
  Are they always mixed up by \(\tilde{R}\)?\\
  \begin{align*}
    T' = \tilde{R} T &= \left( \begin{array}{*{3}{c}}
                                 \multicolumn{1}{l|}{} & &  \\ \cline{1-2}
                                                       & \multicolumn{1}{|l|}{} &  \\ \cline{2-3}
                                                       & &\multicolumn{1}{|l}{} \\
                               \end{array} \right)
    \left( \begin{array}{c}
             \\ \hline
             \\ \hline
             \\
           \end{array} \right)
    \begin{array}{c}
      \}\\
      \}\\
      \}
    \end{array}
    \begin{array}{c}
      \text{elements transform} \\
      \text{among each other}
    \end{array} \\
                     &\text{\textcolor{red}{block diagonal \(\sim\) reducible}}
  \end{align*}
  A block that cannot be reduced any further is an \uwave{irreducible
    representation}.
\item D-dim
  \begin{align*}
    \text{Consider} \ A^{ij}     &= T^{ij} - T^{ji}                                    \\
    \text{Clearly} \ A^{ij}      &= - A^{ji}                                           \\
    A'^{i'j'}                    &= R^{i'i} R^{j'j} T^{ij} - R^{j'i} R^{i'j} T^{ij}    \\
                                 &= R^{i'i} R^{j'j} T^{ij} - R^{j'j} R^{i'i} T^{ji}    \\
                                 &= R^{i'i} R^{j'j} \left( T^{ij} - T^{ji} \right)     \\
                                 &= R^{i'i} R^{j'j} A^{ij}                             \\
    \Rightarrow A^{ij} \         & \text{transform like a tensor!}                     \\
                                 & \text{\uwave{antisymmetric tensor}}                 \\
    \text{With}\                 & \frac{D(D-1)}{2}\ \text{elements}                   \\
                                 &= 3 \quad \text{if}\ D = 3                           \\
    A^{ij} \                     & \text{is reducible}                                 \\ \\
    \text{Consider} \ S^{ij}     &= T^{ij} + T^{ji}                                    \\
    \text{likewise,} \ S'^{i'j'} &= R^{i'i} R^{j'j} S^{ij}                             \\
    \Rightarrow S^{ij} \         & \text{is a \uwave{symmetric tensor.}}               \\
    \text{With}\                 & \frac{D(D+1)}{2} \ \text{elements}                  \\
    \text{But}\ S^{ij}           & \ \text{is again reducible!}                        \\ \\
    \text{Consider}\ S           &= S^{ii}, i \ \text{summed over}                     \\
                         S'      &= R^{i'i} R^{i'j} S^{ij}                             \\
                                 &= \left( R^{\mathrm{T}} \right)^{ii'} R^{i'j} S^{ij} \\
                                 &= \delta^{ij} S^{ij} = S^{ii} = S                    \\
    \Rightarrow S \              & \text{is a scalar and certainly irreducible.}       \\
  \end{align*}
  Now we can get rid of \(S\) from \(S^{ij}\): \\
  let \(\tilde{S}^{ij} = S^{ij} - \delta^{ij} \frac{S}{D}\)
  \begin{exercise}
    how \(\tilde{S}^{ij} is irreducible.\)
  \end{exercise}
  \(\tilde{S}^{ij}\) is a \uwave{symmetric traceless tensor}. \\
  With \textcolor{red}{\(\frac{D(D+1)}{2} - 1\) elements}. \\
  \begin{align*}
    \begin{array}{|c|c||c|c|c|}
      \cline{2-5}
      \multicolumn{1}{l|}{} & T^{ij} & A^{ij} & \tilde{S}^{ij} & S \\ \cline{2-5}
      \multicolumn{1}{l|}{} & D^2 & \frac{D(D-1)}{2} & \frac{D(D+1)}{2} - 1 & 1 \\ \hline
      D=3 & 9 & 3 & 5 & 1 \\ \hline
      D=4& 16 & 6 & 9 & 1 \\ \hline
    \end{array}
  \end{align*}
  \textcolor{red}{in \(D = 3\), \(A^{ij}\) is a vector representation actually! (isomorphic)}
  \begin{align*}
    D = 3,& \\
    & 9 \times 9 \ \text{matrix}\ \tilde{R} = \left( \begin{array}{*{3}{c}}
                                 \multicolumn{1}{l|}{3 \times 3} & &  \\ \cline{1-2}
                                                                 & \multicolumn{1}{|l|}{5 \times 5} &   \\ \cline{2-3}
                                                                 & &\multicolumn{1}{|l}{1 \times 1} \\
                               \end{array} \right)
  \end{align*}
\end{itemize}

\subsection{Restriction to a subgroup}
An irreducible representation of a group \(G\) can be reducible when restricted
to an \(H \subset G\).
\begin{exercise}
  If no time for this, leave it as an exercise to the students.
\end{exercise}

%----------------------------------------------------------------------------------------
%	CHAPTER 2
%----------------------------------------------------------------------------------------
\chapter{Tensors and curved spaces \normalsize{without formal differential geometry}}

\section{Metric}\index{Metric}
Recall the length element in D-dim Euclidean space \\
\begin{align*}
  \dd \tilde{s} &= \delta^{ij} \dd x^i \dd x^j \\
                       & \textcolor{red}{\Downarrow \text{use Gravity letters hereafter}} \\
                       &= \delta^{\mu \nu} \dd x^\mu \dd x^\nu \\
                       &\ \textcolor{red}{\uparrow \text{we call this a \uline{metric}}} \\
\end{align*}
The Euclidean space can have many metrics related by coordinate
tranformations.\\
\begin{example}
  \begin{align*}
    \text{2D plane,\ } & x,y \mapsto r,\theta \\
                       & x = r \cos \theta, y = r \sin \theta \\
    \Rightarrow & \dd x = \cos \theta \dd r - r \sin \theta \dd \theta, \dd y = \sin \theta \dd r + r \cos \theta \dd \theta
  \end{align*}
  \begin{align*}
    \Rightarrow & \dd s^2 = \dd x^2 + \dd y^2 \\
                & = (\cos \theta \dd r - r \sin \theta \dd \theta)^2 + (\sin \theta \dd r + r \cos \theta \dd \theta)^2 \\
                & = \dd r^2 + r^2 \dd \theta^2 \\
                & = g_{\mu\nu} \dd x^\mu \dd x^\nu  \quad \textcolor{red}{x' = r, x^2 = \theta}\\
                & \textcolor{red}{\text{note the upper \& lower indeices}} \\
                & \textcolor{blue}{\left[ g_{\mu\nu} \right] = \mqty(\dmat{1,r^2})}
  \end{align*}
\end{example}
Why upper \& lower? \\
\begin{align*}
  \text{For any vector\ } & v, \abs{v}^2 = \delta^{\mu\nu} v^\mu v^\nu \\
  \text{but} \quad &\left[ \delta^{\mu\nu} \right] v = \mqty(\dmat{1,1}) \mqty(v^1 \\ v^2) = \mqty(v^1 \\ v^2), \\
  \text{however} \quad & \abs{v}^2 = g_{\mu\nu} v^\mu v^\nu \\
                          & \left[ g_{\mu\nu} \right] v = \mqty(\dmat{1,r^2}) \mqty(v^r \\ v^\theta) = \mqty(v^r \\ r^2 v^\theta) \neq v
\end{align*}
\textcolor{red}{It's better to denote \(g_{\mu\nu} v^\nu = v^\mu\)}
\begin{problem}
  There is a catch here! What is it? I will explain shortly.
\end{problem}
\begin{remark}
  \(v_\mu\) comprise the dual vector \(v^*\), linvingin a vector space different
  from that of \(v\). You'd learnt this from a course of diff geometry; however,
  here it's an overkill to distinguish \(v\) from \(v^*\), in order only to
  understand EG.
\end{remark}
\[\textcolor{red}{\Rightarrow \abs{v}^2 = g_{\mu\nu} v^\mu v^\nu = v^\mu v^\mu} \quad
  \textcolor{blue}{\text{called a contradiction of indices.}} \]
\textcolor{red}{can only Einstein-sum a pair of upper \& lower indeices.} \\
This implies \(v_\mu\) can be raised to \(v^\mu\) by the inverse
\begin{align*}
  g^{\mu\nu} v_\nu = v^\mu \\
  g^{\mu\nu}g_{\nu\rho} = \delta^\mu_\rho
\end{align*}
\begin{remark}
  Hereafter, \(\delta\) has both upper \& lower indices \(\xcancel{\delta^{\mu\nu}}, \xcancel{\delta_{\mu\nu}}\) unless in special cases.
\end{remark}
\textcolor{red}{Contracted index is a dummy index: \(v_\mu v^\mu = v_\nu v^\nu\).}

\section{Curved space from change of coordinates}

\subsection {3D euclidean space}
\begin{align*}
  \dd s^2 &= \dd x^2 + \dd y^2 + \dd z^2 \\
          & \downarrow \text{spherical coordinates} \\
          &= \dd r^2 + r^2 \dd \theta^2 + r^2 \sin ^2 \theta \dd \phi^2 \\
          & \downarrow \text{take constant\ } r = a \\
          &= a^2 \textcolor{red}{\underbrace{\textcolor{black}{\left( \dd \theta^2 + \sin^2 \theta \dd \phi^2 \right)}}} \\
          & \text{\textcolor{red}{the length element on a sphere! curved!}}
\end{align*}
\begin{remark}
  This is no big difference between curved spaces (spacetime) and your familiar
  change of coordinates. So, nothing to be afraid of.
\end{remark}
But mathematically, how can you tell \(\dd s^2 = \dd \theta^2 + \sin^2 \theta
\dd \phi^2\) describes a curved space? This is where we need \uwave{differential
geometry}: \textcolor{red}{Given a metric, we can compute the curvature.}
\begin{figure*}[h]
  \centering
  \includestandalone{Pictures/Four-cities}
\end{figure*}
\begin{remark}
  Distance help you tell the curvature. The metric \(g_{\mu\nu}\) tells you
  infinitesimal distances.
\end{remark}

\section{General coordinates transformation}
The equivalence principle says that our spacetime is locally flat, or actually
any curved space is locally flat. The locally flatness can be shown by
coordinate transformation. In fact, the equivalence principle implies that
coordinates do not have an intrinsic significance in geometry and physics. But
since we can freely change coordinates, in a general curved space, how do we
actually do the transformation, and how do the various quantities such as
vectors and tensors are defined and transformed under change of coordinates?
Recall what we did for rotations in \(D\)-dim Euclidean space.

%----------------------------------------------------------------------------------------
%	BIBLIOGRAPHY
%----------------------------------------------------------------------------------------

\chapter*{Bibliography}
\addcontentsline{toc}{chapter}{\textcolor{ocre}{Bibliography}}
\section*{Books}
\addcontentsline{toc}{section}{Books}
\printbibliography[heading=bibempty,type=book]
\section*{Articles}
\addcontentsline{toc}{section}{Articles}
\printbibliography[heading=bibempty,type=article]

%----------------------------------------------------------------------------------------
%	INDEX
%----------------------------------------------------------------------------------------

\cleardoublepage
\phantomsection
\setlength{\columnsep}{0.75cm}
\addcontentsline{toc}{chapter}{\textcolor{ocre}{Index}}
\printindex

%----------------------------------------------------------------------------------------

\end{document}

% Local Variables:
% TeX-command-extra-options: "-shell-escape"
% End:
