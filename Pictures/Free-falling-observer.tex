\documentclass{tikzstandalone}
\begin{document}
\begin{tikzpicture}[line width=2pt, blue,decoration={random steps, amplitude=.5pt,segment length=15pt}]
  % arrowstyle
  \tikzset{>={Straight Barb[length=8pt,width=8pt]}}
  % axes
  % \draw[help lines] (0,0) grid (10,10);
  \coordinate [label=left:\Large\textit{y'}] (y') at (0,8);
  \draw [<-,decorate] (y') -- (0,-1pt);
  \draw [decorate]
  [postaction={decorate,draw,line width=.8pt,decoration={border,amplitude=.6cm,segment length=8pt}}] (10,0) -- (0,0);
  % h,y point
  \coordinate [label=left:\Large\textit{h}] (h) at (0,4);
  % y axis
  \coordinate [label=left:\Large\textit{y}] (y) at (4,6);
  \draw [<-,decorate] (y) -- ++(0,-2);
  \draw [dashed,decorate] (h) -- ++ (4,0);
  % elevator
  \draw [decorate] (h) ++ (3,0) rectangle ++(4,3);
  % ball
  \coordinate (ball) at (5.5,5.5);
  \filldraw (ball) circle [radius=.1cm];
  \draw [->,decorate] (ball) -- ++(0,-.8);
  \coordinate [label=right:\Large{a ball in the elevator}] (ballcom) at (7.5,6.8);
  \draw [decorate]
  [postaction={decorate,draw,decoration={bent,aspect=.3}}] (ball) ++ (0.2,0.2) -- (ballcom);
  % acceleration
  \draw [->,decorate] (7.5,5) -- ++(0,-2.5);
  \coordinate [label=right:\Large{freely-falling elevator}] (accecom) at (7.5,4);
  % equation
  \node (heq) at (5,2) {\huge\(h-\frac{1}{2}gt^2\)};
  \draw [->,decorate] (5,3) -- (5,4);
  \draw [->,decorate] (5,1) -- (5,0);
  % comment
  \node (y'com) at (9,2.2) [label=right:\Large\textcolor{red}{\textit{y'} : observer rest on the ground}] {};
  \node (ycom) at (9,1.2) [label=right:\Large\textcolor{red}{\textit{y}: observer in the elevator}] {};
\end{tikzpicture}
\end{document}
% Local Variables:
% TeX-engine: luatex
% End:
